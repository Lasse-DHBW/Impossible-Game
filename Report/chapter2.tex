\chapter{Theoretische Grundlagen}
\label{chapter:2}

\section{Grundlagen von NEAT}
(2 Seiten)
---- Historisches Wachsen von genetischen Algos - erste ML powered Chatbots, Neuronale Netze, Supervised Learning, Big Data und am Ende das unsupervised deep-learning mit sich selbst adjustierenden Gewichte. Darauf basierend im nächsten Schritt modellierung natürlichen Verhaltens durch genetik. 
---- Bestandteile, N=Neuro-NeuronaleNetze, E=Evolution-Mutation, Crossover, Selection, A=Augmenting- kleinstes NN am Anfang-falls zu schlecht mehr komplexität, T=Topology-Anzahl an Neuronen und Layer sowie Aufbau dessen. Phenotype (ein neuronales Netz), Genotype (ein gewisses Genset), Speciation. Ein Genom mit mehreren Genen, entweder Knoten, oder Kanten-gen. Kantengene haben Innovationsnummern-beschreibt wann eine Kante kreiert wurde, diese werden über alle Netzwerke hinweg gleich sein-wenn ein Netzwerk eine neue Kante erstellt die keine vorheriges hatte bekommt das eine hohe Innovationsnummer. Falls die woanders schon existiert wird die Nummer von der Kante reduziert-Grundlage für gutes Crossover-das passiert nur auf Kantengene, da die Knotengene von dieses abgeleitet werden können-eine Knotenkanten klasse besitzt typischerweise fünf Attribute: 1. Innovationsnummer, 2. Ursprungsknoten, 3. Endknoten, 4. Gewicht, 5. IstAktiv. Ein Knoten hat zwei Attribute: 1. Knotennummer und 2. Knotentyp (Sensor, Hidden, Output) Unterscheidung von excess (am Ende, werden nur vererbt wenn das Elternteil höhere Scores trifft) und disjoint gene (werden immer vererbt und gemerged). DAS WARS ZU CROSSOVER. Fünf Mutations typen, entweder eine Verbindung mutieren das heißt eine neue Verbindung mit dem Gewicht zwischen [-2;2], zweitens man fügt einen Knoten irgendwo dazu auf eine zufällige Verbindung, vorherige Verbidnung behält ihren Wert, die neue Verbindung bekommt Gewicht 1. Drittens eine Verbindung aktivieren, oder deaktivieren. Viertens die Gewichtsverschiebung, multipliziert ein zufälliges Gewicht mit einer Nummer zwischen [0;2]. Fünftens die Gewichtsneusetzung, komplett neues Gewicht zwischen [-2;2]. ENDE MUTATION. Selektion: Unterschiedliche Genome und jedes davon hat einen bestimmten Score in der Simulation erreicht. Speziefizierung  der Genome-Wenn die Distanz zweier Genome unterhalb einer gewissen Grenze ist, gruppiert man diese in eine Spezies - geht nicht um den Score sondern um die Gewichte, wenige Excess und Disjoint Gene. Falls die Distanz zu groß ist und es keine Spezies dafür gibt, erstellt man eine neue Spezies startend mit diesem Genom. Sortieren der Genome innerhalb der einzelnen Spezien basierend auf ihrem Score, absteigend. Dann selektiert man basierend auf einem Schwellenwert eine gewisse Anzahl an Genomen aus jeder Spezie (häufig 50\%) und vernachlässigt alle restlichen im weiteren Programmablauf. Auch sehr schlechte Genome können diesen Prozess durchstehen, da Sie in einer kleinen Spezie sind. Das garantiert Innovation auf der Topologie Ebene, welches durch kleiner Mutationen vielleicht noch wesentlich besser werden könnte. (Grafik zum Grundlegenden Konzept von NEAT)

\section{Grundlagen von simplen Computerspielen}
(1 Seiten)
---- Historische Bedeutung von Computerspielen (hohe gesellschaftliche Durchdringung von Spielen egal ob Militär oder Familie, Kowmpetitv aber begrenzt, zuerst Brettspiele dann Computer, soziales Tun, Vorfornt in der Analyse von menschlichen psychologischen Verhalten)
---- Bestandteile (variierende Umgebung, neue Input Reize, ein Ziel, z.B. Highscore oder Überleben, häufig Kompetitiv, Spieler steuert einen Agenten in einer simplifizierten Welt, visuell häufig ausgeklügelt aber nicht ausschlaggebend für gutes Spiel -> Überleitung zu den Grundlagen der Visualisierung)

\section{Grundlagen zu Visualisierungen}

Bei Visualisierungen handelt es sich um ein vielschichtiges und multidisziplinäres Thema. Dementsprechend zollen unterschiedliche Disziplinen dem Thema Respekt zu auf diversen unterschiedlichen Ebenen, sowohl was fachliche tiefe als auch Domänenwissen angeht. Dabei basieren sie alle auf der biologische Grundlage der Informationsverarbeitung durch Sehen und der visuellen Perzeption. 

\textbf{Neurobiologie der Informationsverarbeitung}

Hierfür gibt es eine viel zahl an psychologischen Modellen z.B. Behaviorismus, Mehrspeichermodell welche die Informationsverarbeitung auf unterschiedlichen Ebenen versuchen zu generalisieren.  Sie alle gelten jedoch nicht als vollständig akkurat und sollten kritisch betrachtet werden. \cite{TUD-V1} Die zentralen und offenen Forschungsfragen der Gedächtnisforschung gelten hierbei grundlegenden Systeme, Repräsentationen, Prozesse und das Neurobiologische Substrat ab. \cite[S. 69]{TUD-VL01}

https://tu-dresden.de/mn/psychologie/ifap/allgpsy/ressourcen/dateien/lehre/lehreveranstaltungen/goschke\_lehre/ws\_2013/vl\_gedaechtnis/V-Mehrspeichermodell.pdf?lang=de

Weitestgehender Konsens besteht allerdings im Beginn des Prozesses der Informationsverarbeitung, welcher häufig mit den sensorischen Rezeptoren des Menschen, welche die Umgebung in elektrochemische Signale umwandelt beginnt. Diese Umwandlung, bekannt als sensorische Transduktion, erfolgt, wenn ein Stimulus von einem Rezeptor erkannt wird, der ein graduiertes Potential in einem sensorischen Neuron erzeugt. Wenn dieses Potential stark genug ist, erzeugt das sensorische Neuron ein Aktionspotential, das ins zentrale Nervensystem weitergeleitet wird, wo es mit anderen sensorischen Informationen integriert wird, um eine bewusste Wahrnehmung dieses Stimulus zu ermöglichen und eine Handlung auszuführen \cite{BasicHumanPhysiology, AnatomyPhysiology}. Hierbei agieren die acht menschlichen Sinne als unterschiedlich spezialisierte, sensorischen, Rezeptoren für die jeweils eine separate Transduktion erfolgt, welche es gilt zusammenzuführen. Dafür maßgeblich verantwortlich ist das limbische System, insbesondere der Hippocampus und die Amygdala. Diese spielen eine zentrale Rolle bei der Bewertung und Speicherung von Informationen. Es vergleicht eingehende sensorische Daten mit vorhandenem Wissen und Emotionen, was entscheidend für die Gedächtnisbildung ist \cite{WÜRZ, UllmannLimbicSystem}.

\textbf{Lernen im Kontext der Informationsverarbeitung}

Die Gedächtnisbildung auch genannte Lernen, bezeichnet Prozesse die dem erfahrungsabhängigen Erwerb von Wissen oder Fertigkeiten sowie der Veränderung von Verhaltendsdispositonen zugrunde ligen.
Darauf aufbauen bezeichnet das Gedächtnis selbst die Ergebnisse des Lernens wie zum Beispiel Erinnerungen, Wissen und Fertigkeiten. 
Abgrenzung erfahren diese beiden Gebiete von Dingen wie Verhaltensveränderung die nicht auf das Lernen zurückzuführen sind (Drogen), aber auch der genetische angelegte Veränderungsprozess des Nervensystems namens Reifung und zuletzt auch die Prägung von instinktiven Verhalten durch Ereignissen in kritischen Lebensphasen.\cite[S. 16-18]{TUD-VL01}

tps://tu-dresden.de/mn/psychologie/ifap/allgpsy/ressourcen/dateien/lehre/lehreveranstaltungen/goschke\_lehre/ws2014/ppt\_lernen\_ged2014/VL01-Einfuehrung.pdf?lang=en

Aus neurobiologischer Perspektive bedeutet Lernen einen ständigen Aufbau
von Neuronenpopulationen im Cortex. Jedes Neugeborene kommt mit ca.
100 Milliarden Neuronen auf die Welt. Diese sind jedoch nur sehr lose
miteinander verknüpft. Im ersten Lebensjahr vergrößert das Baby seine
Gehirnmasse von ca. 250 g auf 750 g. Dies geschieht nur dadurch, dass das
Baby “lernt“. Es entstehen feste Verbindungen zwischen den Neuronen,
sodass es zu Neuronenpopulationen kommt.
Gegenstand der Betrachtung sind die Stoffwechselprozesse im Gehirn sowie
die Wirkungsweise der Botenstoffe (Neurotransmitter). Hierdurch werden
bekannte Vorgehensweisen (Handlungsorientierung, positives Feedback,
Wechsel der Sozialformen, Prinzip der Wiederholung …) bestätigt und neue
Erkenntnisse gewonnen.

Darauf basierend stellt sich die naheliegende Frage, welche von diesen Sinnen den größten Einfluss auf die am Ende wahrgenommene Situation und ausgeführte Handlung hat. 


\textbf{Visualisierung im Lernkontext}

Visualisierungen sind besonders effektiv im Lernprozess, da sie verschiedene kognitive Ebenen stimulieren und so das Verständnis fördern. Sie ermöglichen eine tiefere und umfassendere Verarbeitung von Informationen, indem sie komplexe Konzepte in einer zugänglichen und intuitiven Form darstellen. Dies ist besonders wichtig, da die Neurobiologie des Lernens betont, wie entscheidend es ist, Lerninhalte in einer Weise zu präsentieren, die mit der Funktionsweise des menschlichen Gehirns harmoniert \cite{UllmannMentalRepresentation}.

\textbf{Integration verschiedener Sinnesreize}

Interessant ist der Vergleich der Verarbeitungskapazitäten zwischen visuellen und auditiven Rezeptoren. Die Integration auditiver Stimuli kann die sensorische Kodierung in den Dendriten von Pyramidenzellen im somatosensorischen Cortex verstärken, was die Verarbeitung von Berührungsreizen beeinflusst \cite{NatureCommunications}. Diese multisensorische Integration unterstreicht die Fähigkeit des Gehirns, Informationen aus verschiedenen sensorischen Pfaden effizient zu verarbeiten.

\textbf{Forschungsperspektiven und Zukunft der Visualisierung}

Die Forschung in den Neurowissenschaften, insbesondere die simultanen Aufzeichnungen von Neuronenpopulationen, bietet wertvolle Einblicke in die Varianz der neuronalen Antworten. Solche Erkenntnisse sind entscheidend für das Verständnis, wie das Gehirn lernt und Informationen verarbeitet \cite{FrontiersInNeuroscience}. Zukünftige Forschungen und Entwicklungen im Bereich der Visualisierung versprechen, die Effektivität des Lernens weiter zu steigern, indem sie noch stärker auf die neurobiologischen Grundlagen des Lernens abgestimmt werden.


---- Ganz Basic warum Visualisierung -> weil leicht konsumierbar von Menschen, häufig leichter verständlicher und kompensierter als Text. 

https://open.oregonstate.education/aandp/chapter/13-1-sensory-receptors/
https://www.uni-wuerzburg.de/fileadmin/43060000/04\_Fort-\_und\_Weiterbildungen\_Lehrkraefte/Herbsttagungen/Herbsttagung\_2016/20161006\_WS\_04\_Neurobiologie.pdf
https://www.frontiersin.org/articles/10.3389/fncir.2015.00051/full
https://www.ncbi.nlm.nih.gov/pmc/articles/PMC6091269/


---- Psychologische Effekte von Visualisierung (vgl. Text/Bild/Videos) in spielen, dann verallgemeinert auch beim lernen, leichteres lernen? schnelleres lernen? (Grafik zum biologischen sehen) 
---- Neuster Stand bei visuellen Effekten und Erkenntnisse, Mikroanimationen, Zerlegung komplexer Themen in Videos - Bewegtbildproduktionen. Trend zu immer kürzeren Videos und schnellere Schnittfolge um Aufmerksamkeit zu bewahren. Finder Anklang auch in der Wissenschaft, leichtere Vermittlung komplexer Inhalte an viele Zuhörer - mehr Wissensvideos zum Beispiel auch im Informatik/Mathematik und Biologie Bereich. 

Visualisierungen erhöhen den Lernerfolg und reduzieren die mentale Belastungen bei dem Verständnis von wissenschaftlichen Texten. (vgl. S. 118 d-nb.info)
Zudem reduzieren Videos die mentale Belastung im Vergleich zu statischen Bildern beim Verständnis neuer Konzepte. (vgl. S. 324 hbz-nrw.de)


