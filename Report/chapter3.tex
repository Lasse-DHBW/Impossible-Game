% !TEX root = master.tex
\chapter{Praktische Ausarbeitung}
\label{chapter:3}
\section{Projektbeschreibung}
Das Ziel dieses Projekts, inspiriert durch die Faszination für die Modellierung natürlicher Ereignisse in computergesteuerten Systemen, ist die Entwicklung eines Agenten, basierend auf dem NEAT-Algorithmus, um ein Computerspiel zu meistern.

Für das Projekt wurde ein Standardspiel aus der Gymnasium (gym) Bibliothek ausgewählt \cite{gymnasiumbib}. Gym, ein Open-Source-Python-Paket entwickelt von OpenAI, bietet eine Sammlung standardisierter Umgebungen zur Entwicklung und zum Testen von Algorithmen im Bereich des Reinforcement Learning (RL). Es erleichtert die Entwicklung und Vergleichbarkeit von RL-Algorithmen durch eine einheitliche Schnittstelle für unterschiedliche Umgebungen. Der NEAT-Algorithmus eignet sich zur Bewältigung von RL-Aufgaben, weshalb diese Bibliothek Anwendung findet. Gym umfasst zahlreiche Simulationsumgebungen, einschließlich aller Atari-Spiele.

Das ausgewählte Spiel, \enquote{Lunar Lander}, zeichnet sich durch ein einfaches Ziel und einfache Steuerungselemente aus. Der Inhalt des Spiels besteht darin, einen Mondlander sicher und waagrecht in einem markierten Bereich auf der Mondoberfläche zu landen. Die Leistungskriterien umfassen den waagrechten Anflug, den Treibstoffverbrauch und ob der Lander in der richtigen Orientierung sicher im Zielbereich landet. Zur Steuerung des Landers werden zwei lineare Signale $a$ und $b$ verwendet, um die Leistung des Haupt- und der Seitentriebwerke zu regulieren. Das Haupttriebwerk wird bei $a>0,5$ aktiviert, wobei die Intensität bis $a=1$ proportional zunimmt. Ähnlich feuern das linke oder rechte Triebwerk bei $-1\leq b<-0,5$ bzw. $1 \geq b>0,5$.

Das technische Ziel des Projekts ist es, eine Konfiguration für ein neuronales Netz zu finden, sodass basierend auf der relativen Position des Landers die Triebwerke so gesteuert werden, dass eine Landung möglichst waagrecht, mit geringem Treibstoffverbrauch und innerhalb des Zielbereichs erfolgt.

Anschließend wird ein Video konzipiert und erstellt, das den NEAT-Algorithmus erklärt und am Beispiel des Lunar Landers veranschaulicht.

\section{Implementierungskonzept}
Für die Realisierung dieses Projektes wurde die Programmiersprache Python ausgewählt. In Python wurden sowohl die einzelnen Komponenten des NEAT-Algorithmus als auch dessen Integration mit dem ausgewählten Spiel umgesetzt. Die benötigte Simulationsumgebung für das Spiel wird durch die Bibliothek Gymnasium bereitgestellt \cite{gymnasiumbib}.

Als funktionale Anforderungen wurden folgende Punkte festgelegt:
\begin{itemize}
	\item Der NEAT-Algorithmus soll basierend auf dem Whitepaper implementiert und getestet werden.
	\item Der Lander soll ausschließlich durch einen Agenten gesteuert werden, der auf dem NEAT-Algorithmus basiert.
	\item Alle Aktionen des Landers sollen aus den Beobachtungsvariablen des 8-dimensionalen Inputvektors abgeleitet werden.
	\item Der finale Agent soll als Lösung mindestens 200 Punkte erreichen \cite{gymnasiumdoc}.
\end{itemize}

Als nicht-funktionale Anforderung wurde festgelegt:
\begin{itemize}
	\item Das Programm zur Ermittlung eines optimalen Landers soll einfach und replizierbar sein.
\end{itemize}

\textbf{Implementierungsdetails}

Die Implementierung des NEAT-Algorithmus folgt der ursprünglichen Beschreibung \cite{NEAT}. An einigen Stellen wurde die Vorgehensweise der existierenden Bibliothek Python NEAT \cite{pythonneat} als Orientierung genutzt, beispielsweise bei der zufälligen Generierung von Gewichtswerten mittels Normalverteilung. Diese Aspekte sind in der originalen Beschreibung nicht ausführlich erläutert, sondern nur im dazugehörigen C++-Code zu finden.

\section{Visualisierungskonzept}

Das Visualisierungskonzept dieses Projektes basiert auf den Vorgaben des Kurses. Es wurde Python in Kombination mit Manim verwendet, um das Konzept ansprechend in Videoform zu animieren und darzustellen. Die Vertonung erfolgte durch das Text-to-Speech-Programm ElevenLabs, und der Schnitt sowie die Videobearbeitung wurden mit ClipChamp durchgeführt.

Die Anforderungen an diesen Projektteil unterteilen sich in funktionale und nicht-funktionale Kriterien:

Funktionale Anforderungen:
\begin{itemize}
	\item Animation und Rendering des Videos mit Python und Manim.
	\item Die Videolänge überschreitet 15 Minuten nicht.
	\item Vollständige Vertonung des Videos.
\end{itemize}

Nicht-funktionale Anforderungen:
\begin{itemize}
	\item Ansprechende Animation.
	\item Vollständige und verständliche Darstellung des NEAT-Konzepts.
	\item Berücksichtigung von Visualisierungsgrundlagen.
	\item Ein klar erkennbarer roter Faden.
\end{itemize}

Bei der Umsetzung war es wichtig, die erarbeiteten Visualisierungsgrundlagen effektiv einzusetzen. Ziel war es, den NEAT-Algorithmus mittels natürlicher Phänomene insbesondere für technisch versierte MINT-Studenten verständlich und inspirierend zu erklären.

In der visuellen Gestaltung lag der Fokus auf darstellenden und organisierenden Bildern, um den Informationsfluss zu strukturieren. Interpretierende Bilder kamen zum Einsatz, um komplexe Theorien zu vermitteln und die Aufmerksamkeit der Zuschauer zu gewinnen. Als Medium wurde das Video gewählt, wobei die Lesbarkeit durch die Schriftart Montserrat und helle Schrift auf dunklem Hintergrund unterstützt wurde. Ergänzt wurde dies durch Umrandungen, Hervorhebungen und gezielten Farbeinsatz. Die Präsentation der Daten erfolgte hauptsächlich über Diagramme, welche die neuronalen Netze abbildeten.

Die Planung des Projekts begann mit einem skizzierten Storyboard, das von allen Gruppenmitgliedern diskutiert und verfeinert wurde. Nach der Finalisierung erfolgte die schrittweise Umsetzung und Animation. Die Vertonung wurde abschnittsweise durchgeführt.

 