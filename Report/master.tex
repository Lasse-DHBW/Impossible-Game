%%%%%%%%%%%%%%%%%%%%%%%%%%%%%%%%%%%%%%%%%%%%%%%%%%%%%%%%%%
%   Autoren:
%   Prof. Dr. Bernhard Drabant
%   Prof. Dr. Dennis Pfisterer
%   Prof. Dr. Julian Reichwald
%%%%%%%%%%%%%%%%%%%%%%%%%%%%%%%%%%%%%%%%%%%%%%%%%%%%%%%%%%

%%%%%%%%%%%%%%%%%%%%%%%%%%%%%%%%%%%%%%%%%%%%%%%%%%%%%%%%%%
%	ANLEITUNG: 
%   1. Ersetzen Sie firmenlogo.jpg im Verzeichnis img
%   2. Passen Sie alle Stellen im Dokument an, die mit 
%      @stud 
%      markiert sind
%%%%%%%%%%%%%%%%%%%%%%%%%%%%%%%%%%%%%%%%%%%%%%%%%%%%%%%%%%

%%%%%%%%%%%%%%%%%%%%%%%%%%%%%%%%%%%%%%%%%%%%%%%%%%%%%%%%%%
%	ACHTUNG: 
%   Für das Erstellen des Literaturverzeichnisses wird das 
%   modernere Paket biblatex in Kombination mit biber 
%   verwendet - nicht mehr das ältere Paket BibTex!
%
%   Bitte stellen Sie Ihre TeX-Umgebung entsprechend ein (z.B. TeXStudio): 
%   Einstellungen --> Erzeugen --> Standard Bibliographieprogramm: biber
%%%%%%%%%%%%%%%%%%%%%%%%%%%%%%%%%%%%%%%%%%%%%%%%%%%%%%%%%%

\documentclass[fontsize=12pt,BCOR=5mm,DIV=12,parskip=half,listof=totoc,
               paper=a4,toc=bibliography,pointlessnumbers]{scrreprt}
               
\makeindex

%% Elementare Pakete, Konfigurationen und Definitionen werden geladen (gegebenenfalls anpassen)
% !TEX root =  master.tex

%%%%%%%%%%%%%%%%%%%%%%%%%%%%%%%%%%%%%%%%%%%%%%%%%%%%%%%%%%%%%%%%%%
%	ANLEITUNG: 
% Passen Sie gegebenenfalls alle Stellen im Dokument an, die mit 
% @stud 
% markiert sind.
%%%%%%%%%%%%%%%%%%%%%%%%%%%%%%%%%%%%%%%%%%%%%%%%%%%%%%%%%%%%%%%%%%

%%
%% @stud
%%
%% LANGUAGE SETTINGS
\usepackage[ngerman]{babel} 	        % german language
\usepackage[german=quotes]{csquotes} 	% correct quoting using \enquote{}
%\usepackage[english]{babel}          % english language
%\usepackage{csquotes} 	              % correct quoting using \enquote{}
\usepackage{eurosym}
\usepackage{makeidx}                  % allows index generation
\usepackage{listings}	                %Format Listings properly
\usepackage{lipsum}                   % Blindtext
\usepackage{graphicx}       
\usepackage{tabularx}                          % use various graphics formats
\usepackage[german]{varioref}
\usepackage{caption}	                % better Captions
\usepackage{booktabs}                 % nicer Tabs
\usepackage[hidelinks=true]{hyperref} % keine roten Markierungen bei Links
\usepackage{fnpct}                    % Correct superscripts 
\usepackage{calc}                     % Used for extra space below footsepline, in particular
\usepackage{array}
\usepackage{acronym}
\usepackage{algorithm}
\usepackage{algpseudocode}
\usepackage{setspace}
\usepackage{tocloft}
\usepackage{makecell}
\usepackage{hhline}

\usepackage{longtable}
\usepackage{svg}
\usepackage{amssymb}

%Benedikt
\usepackage{xltabular}

% Davids Abschnitt
\usepackage{geometry}
\usepackage{pdflscape}
\usepackage{wrapfig}

%% Schriftarten- und Zeichenpakete
\usepackage[T1]{fontenc}
\usepackage[utf8]{inputenc}

%%
%% @stud
%%
%%	FONT SELECTION: Schriftarten und Schriftfamilie
%%%%%%%%%%%%%
%% SCHRIFTART
%%%%%%%%%%%%%
% 0) without decomment: normal font families 
% ...
% 1) Latin Modern 
\usepackage{lmodern}        
% 2) Times 
%\usepackage{mathptmx}         
% 3) Helvetica
%\usepackage[scaled=.92]{helvet} 
%%%%%%%%%%%%%%%%%%
%%	SCHRIFTFAMILIE
%%%%%%%%%%%%%%%%%%
% ohne Serifen
\renewcommand*{\familydefault}{\sfdefault}
\addtokomafont{disposition}{\sffamily}
%
% mit Serifen
%\renewcommand*{\familydefault}{\rmdefault}
%\addtokomafont{disposition}{\rmfamily}
%
% Typewriter
%\renewcommand*{\familydefault}{\ttdefault}
%\addtokomafont{disposition}{\ttfamily}

%%
%% @stud
%%
%% Uncomment the following lines to support hard URL breaks in bibliography 
%\apptocmd{\UrlBreaks}{\do\f\do\m}{}{}
%\setcounter{biburllcpenalty}{9000}% Kleinbuchstaben
%\setcounter{biburlucpenalty}{9000}% Großbuchstaben

%%
%% @stud
%%
%% FOOTNOTES: Count footnotes over chapters
%% \counterwithout{footnote}{chapter}

%	ACRONYMS
\makeatletter
\@ifpackagelater{acronym}{2015/03/20}
{\renewcommand*{\aclabelfont}[1]{\textbf{{\acsfont{#1}}}}}{}
\makeatother

%	LISTINGS
% @stud: ggf. Namen/Text anpassen (englisch)
\renewcommand{\lstlistingname}{Quelltext} 
\renewcommand{\lstlistlistingname}{Quelltextverzeichnis}
\lstset{numbers=left,
	numberstyle=\tiny,
	captionpos=b,
	basicstyle=\ttfamily\small}

%	ALGORITHMS
% @stud: ggf. Namen/Text anpassen (englisch)
\renewcommand{\listalgorithmname}{Algorithmenverzeichnis}
\floatname{algorithm}{Algorithmus}

%	PAGE HEADER / FOOTER
%	Warning: There are some redefinitions throughout the master.tex-file!  DON'T CHANGE THESE REDEFINITIONS!
\RequirePackage[automark]{scrlayer-scrpage}
%alternatively with separation lines: \RequirePackage[automark,headsepline,footsepline]{scrlayer-scrpage}

\renewcommand{\chaptermarkformat}{}
\RedeclareSectionCommand[beforeskip=0pt]{chapter}
\clearpairofpagestyles

%\ifoot[\rule{0pt}{\ht\strutbox+\dp\strutbox}DHBW Mannheim]{\rule{0pt}{\ht\strutbox+\dp\strutbox}DHBW Mannheim}
\ofoot[\rule{0pt}{\ht\strutbox+\dp\strutbox}\pagemark]{\rule{0pt}{\ht\strutbox+\dp\strutbox}\pagemark}
\ohead{\headmark}

\newcommand{\TitelDerArbeit}[1]{\def\DerTitelDerArbeit{#1}\hypersetup{pdftitle={#1}}}
\newcommand{\AutorDerArbeit}[1]{\def\DerAutorDerArbeit{#1}\hypersetup{pdfauthor={#1}}}
\newcommand{\Firma}[1]{\def\DerNameDerFirma{#1}}
\newcommand{\Kurs}[1]{\def\DieKursbezeichnung{#1}}
\newcommand{\Abteilung}[1]{\def\DerNameDerAbteilung{#1}}
\newcommand{\Studiengangsleiter}[1]{\def\DerStudiengangsleiter{#1}}
\newcommand{\WissBetreuer}[1]{\def\DerWissBetreuer{#1}}
\newcommand{\FirmenBetreuer}[1]{\def\DerFirmenBetreuer{#1}}
\newcommand{\Bearbeitungszeitraum}[1]{\def\DerBearbeitungszeitraum{#1}}
\newcommand{\Abgabedatum}[1]{\def\DasAbgabedatum{#1}}
\newcommand{\Matrikelnummer}[1]{\def\DieMatrikelnummer{#1}}
\newcommand{\Studienrichtung}[1]{\def\DieStudienrichtung{#1}}
\newcommand{\ArtDerArbeit}[1]{\def\DieArtDerArbeit{#1}}
\newcommand{\Literaturverzeichnis}{Literaturverzeichnis}

\newcommand{\DavidsGeileTabellen}[1]{
	\pagebreak
	
	\newgeometry{left=1cm,right=1cm,top=1cm,bottom=1cm}
	\begin{landscape}

		
		\thispagestyle{empty}
		
		\begin{table}
			\begin{tabularx}{27.7cm}{c c p{3cm} X c c c p{2.5cm} c X}
				\textbf{ID} &\textbf{Risikotyp}&\textbf{Titel} &\textbf{Beschreibung} &\textbf{WSK} &\textbf{Stärke} &\textbf{Risiko} &\textbf{Problem Zeitpunkt} &\textbf{MKat} &\textbf{Maßnahme}\\\toprule
				#1
			\end{tabularx}
		\end{table}


	\end{landscape}
	\restoregeometry
}

\newcommand{\settingBibFootnoteCite}{
	\setlength{\bibparsep}{\parskip}		  % Add some space between biblatex entries in the bibliography
	\addbibresource{bibliography.bib}	    % Add file bibliography.bib as biblatex resource
	\DefineBibliographyStrings{ngerman}{andothers = {{et\,al\adddot}},}
}

\newcommand{\setTitlepage}{
	% !TEX root =  master.tex
% @stud: ggf. Namen/Text anpassen (englisch)
\begin{titlepage}
\begin{minipage}{\textwidth}
		\vspace{-2cm}
		\noindent \hfill \includegraphics{\imagedir/logo.jpg}
\end{minipage}
\vspace{1em}
%\sffamily
\begin{center}
	{\textsf{\large Duale Hochschule Baden-W\"urttemberg Mannheim}}\\[4em]
	{\textsf{\textbf{\large{\DieArtDerArbeit}arbeit}}}\\[6mm]
	{\textsf{\textbf{\Large{}\DerTitelDerArbeit}}} \\[1.5cm]
	{\textsf{\textbf{\large{}Studiengang Wirtschaftsinformatik}}\\[6mm]
	\textsf{\textbf{Studienrichtung \DieStudienrichtung}}}\vspace{15em}
	
	\begin{minipage}{\textwidth}
		\begin{tabbing}
		Wissenschaftliche(r) Betreuer(in): \hspace{0.85cm}\=\kill
		Verfasser: \> \DerAutorDerArbeit \\[1.5mm]
		Matrikelnummer: \> \DieMatrikelnummer \\[1.5mm]
%		Firma: \> \DerNameDerFirma  \\[1.5mm]
%		Abteilung: \> \DerNameDerAbteilung \\[1.5mm]
		Kurs: \> \DieKursbezeichnung \\[1.5mm]
		Studiengangsleiter: \> \DerStudiengangsleiter \\[1.5mm]
		Dozent: \> \DerWissBetreuer \\[1.5mm]
%		Firmenbetreuer(in): \> \DerFirmenBetreuer \\[1.5mm]
		Bearbeitungszeitraum: \> \DerBearbeitungszeitraum\\[1.5mm]
%		alternativ:\\[1.5mm]
%		Eingereicht: \> \DasAbgabedatum	
		\end{tabbing}
	\end{minipage}
\end{center}
\end{titlepage}
	\pagenumbering{roman} % Römische Seitennummerierung
	\normalfont	
}

\newcommand{\initializeText}{
	\clearpage
	\ihead{\chaptername~\thechapter} % Neue Header-Definition
	\pagenumbering{arabic}           % Arabische Seitenzahlen
}

\newcommand{\initializeBibliography}{
	\ihead{}
	\printbibliography[title=\Literaturverzeichnis] 
	\cleardoublepage
}

\newcommand{\initializeAppendix}{
	\appendix
	\ihead{}
	\cftaddtitleline{toc}{chapter}{Anhang}{}
}


%%
%% @stud
%%
%% PERSÖNLICHE ANGABEN (BITTE VOLLSTÄNDIG EINGEBEN zwischen den Klammern: {...})
%%
\ArtDerArbeit{Gruppen} % "Bachelor" oder "Projekt" wählen
\TitelDerArbeit{Steuerung von Agenten in Videospielen basierend auf dem genetischen Algorithmus NEAT}
\AutorDerArbeit{Amos Dinh, David Schäfer, Lasse Friedrich}
%\Abteilung{<Ihre Abteilung>}
%\Firma{<Ihre Firma>}
\Kurs{WWI-21-DSA}
\Studienrichtung{Data Science}
\Matrikelnummer{5504890, 7086451, 9924680}
\Studiengangsleiter{Prof. Dr.-Ing. habil. Dennis Pfisterer}
\WissBetreuer{Dr. Maximilian Scherer}
%\FirmenBetreuer{<Ihr(e) Firmenbetreuer(in)>}
\Bearbeitungszeitraum{13.11.2023 -- 07.02.2023}

%%
%% @stud
%%
%% BIBLIOGRAPHY (@stud: Bibliographie-Stil wählen - Position und Indizierung)
%%  Auswahl zwischen: 
%%   NUMERIC Style
%%   IEEE Style
%%   ALPHABETIC Style
%%   HARVARD Style
%%   CHICAGO Style 
%%   (oder eigenen zulässigen Stil wählen) 
%%
%%%%%%%%%%%%%
%% Zitierstil
%%%%%%%%%%%%%
% NUMERIC Style - e. g. [12]
\newcommand{\indextype}{numeric} 
%
% IEEE Style - numeric kind of style 
%\newcommand{\indextype}{ieee} 
%
% ALPHABETIC Style - e. g. [AB12]
%\newcommand{\indextype}{alphabetic} 
%
% HARVARD Style 
%\newcommand{\indextype}{apa} 
%
% CHICAGO Style 
%\newcommand{\indextype}{authoryear}
%
%%%%%%%%%%%%%%%%%%%%%%
%% Position des Zitats
%%%%%%%%%%%%%%%%%%%%%%
\newcommand{\position}{inline} 
%
% (!!) FOOTNOTE POSITION NOT RECOMMENDED IN MINT DOMAIN:
%\newcommand{\position}{footnote}

%% Final: Setzen des Zitierstils und der Zitatposition
\usepackage[backend=biber, autocite=\position, style=\indextype]{biblatex} 	
\settingBibFootnoteCite

%%
%% Definitionen und Commands
%%
\newcommand{\abs}{\par\vskip 0.2cm\goodbreak\noindent}
\newcommand{\nl}{\par\noindent}
\newcommand{\mcl}[1]{\mathcal{#1}}
\newcommand{\nowrite}[1]{}
\newcommand{\NN}{{\mathbb N}}
\newcommand{\imagedir}{img}

\makeindex

\begin{document}

\setTitlepage

%%%%%%%%%%%%%%%%%%%%%%%%%%%%%%%%%%%
% EHRENWÖRTLICHE ERKLÄRUNG
%
% @stud: ewerkl.tex bearbeiten
%
%\input{ewerkl} 
%\cleardoublepage  
%%%%%%%%%%%%%%%%%%%%%%%%%%%%%%%%%%%

%%%%%%%%%%%%%%%%%%%%%%%%%%%%%%%%%%%
% SPERRVERMERK
%
% @stud: nondisclosurenotice.tex bearbeiten
%
%% !TEX root =  master.tex
\chapter*{Sperrvermerk}

\begin{center}
\fbox{
		\begin{minipage}{33em}
			\textbf{Ein Sperrvermerk sollte nur bei berechtigtem Bedarf gesetzt werden!\\[10pt] 
				Beachten Sie, dass mit Sperrvermerk	versehene Arbeiten nicht für weitere wissenschaftliche Zwecke 
				außerhalb des Firmenkontextes oder zur Publikation verwendet werden dürfen.\\[10pt]
				Wir empfehlen, wenn m\"oglich, auf den Sperrvermerk zu verzichten.\\[10pt]
				Besprechen Sie diese Problematik mit Ihrer Firma!}
		\end{minipage}
}
\end{center}

(Mustertext) Der Inhalt dieser Arbeit darf weder als Ganzes noch in Auszügen Personen außerhalb des Prüfungsprozesses 
und des Evaluationsverfahrens zugänglich gemacht werden, sofern keine anders lautende Genehmigung der Ausbildungsstätte vorliegt. 

\cleardoublepage
 
%\cleardoublepage
%%%%%%%%%%%%%%%%%%%%%%%%%%%%%%%%%%%

%%%%%%%%%%%%%%%%%%%%%%%%%%%%%%%%%%%
%	KURZFASSUNG
%
% @stud: acknowledge.tex bearbeiten
%
%% !TEX root =  master.tex
\chapter*{Danksagung}

Hier können Sie eine Danksagung schreiben. 



%\cleardoublepage 
%%%%%%%%%%%%%%%%%%%%%%%%%%%%%%%%%%%

%%%%%%%%%%%%%%%%%%%%%%%%%%%%%%%%%%%
% VERZEICHNISSE und ABSTRACT
%
% @stud: ggf. nicht benötigte Verzeichnisse auskommentieren/löschen
%
\tableofcontents
\cleardoublepage

% Abbildungsverzeichnis
\phantomsection
\addcontentsline{toc}{chapter}{\listfigurename}
\listoffigures
\cleardoublepage

%	Tabellenverzeichnis
\phantomsection
\addcontentsline{toc}{chapter}{\listtablename}
\listoftables
\cleardoublepage

%	Listingsverzeichnis / Quelltextverzeichnis
%\lstlistoflistings
%\cleardoublepage

% Algorithmenverzeichnis
%\listofalgorithms
%\cleardoublepage

% Abkürzungsverzeichnis
% @stud: acronyms.tex bearbeiten
%% !TEX root =  master.tex
\clearpage
\chapter*{Abkürzungsverzeichnis}	
\addcontentsline{toc}{chapter}{Abkürzungsverzeichnis}

\begin{acronym}[XXXXXXX]
	\acro{ad}[AD]{Archiv für Diplomatik, Schriftgeschichte, Siegel- und Wappenkunde}
	\acro{BMBF}{Bundesministerium für Bildung und Forschung}	
	\acro{DHBW}{Duale Hochschule Baden-Württemberg}
	\acro{ecu}[ECU]{European Currency Unit}
	\acro{eu}[EU]{Europäische Union}
	\acro{RDBMS}[RDBMS]{Relational Database Management System}
\end{acronym} 
%\cleardoublepage

%	Kurzfassung / Abstract
% @stud: abstract.tex bearbeiten
% !TEX root =  master.tex
\chapter*{Abstract}
\addcontentsline{toc}{chapter}{Abstract}
Dieses Projekt zielt auf die Implementierung und Visualisierung des genetischen Algorithmus NEAT (Neuroevolution of augmenting topologies) ab, angewendet auf das {Lunar Lander}-Spiel aus der Gymnasium-Bibliothek. Der Fokus liegt auf der Entwicklung eines Lehrvideos, das den Algorithmus und seine Anwendung im Spielkontext veranschaulicht. Unter Verwendung der Python-Bibliothek Manim wurde ein detailliertes Visualisierungskonzept entworfen und umgesetzt, um die Funktionsweise von NEAT eingängig zu erklären.

Im Zuge des Projekts wurde der NEAT-Algorithmus von Grund auf implementiert und in das \enquote{Lunar Lander}-Spiel integriert. Durch das vertonte Lehrvideo, das sowohl visuell ansprechend als auch informativ gestaltet wurde, konnte eine klare und verständliche Darstellung des Algorithmus erreicht werden. Das Endergebnis ist ein umfassendes Lehrmittel, das sowohl die technische Implementierung als auch die theoretischen Grundlagen von NEAT effektiv vermittelt.
 
\cleardoublepage

%%%%%%%%%%%%%%%%%%%%%%%%%%%%%%%%%%%%%%%%%%%%%%%%%%%%%%%%%%%%%%%%%%%%%%%%%%%%%%%%%%%%%%%%%%
% KAPITEL UND ANHÄNGE
%
% @stud:
%   - nicht benötigte: auskommentieren/löschen
%   - neue: bei Bedarf hinzufügen mittels input-Komman do an entsprechender Stelle einfügen
%%%%%%%%%%%%%%%%%%%%%%%%%%%%%%%%%%%%%%%%%%%%%%%%%%%%%%%%%%%%%%%%%%%%%%%%%%%%%%%%%%%%%%%%%%

\initializeText
\onehalfspacing

%%%%%%%%%%%%%%%%%%%%%%%%%%%%%%%%%%%
% KAPITEL
%
% @stud: einzelne Kapitel bearbeiten und eigene Kapitel hier einfügen
%
% Einleitung
%% !TEX root =  master.tex
\chapter{Einleitung}

\nocite{*}

Dieses Kapitel enthält die Einleitung mit ihren verschiedenen Abschnitten/Sections und Unterabschnitten.

\section{Beispiel Abschnitt: \LaTeX-Installation}

Zur Verwendung von \LaTeX-Installation einer Distribution z.~B.~TeXLive, MikTex etc.~sowie eines Editors z.~B.~TeXStudio, TeXnicCenter etc.~notwendig.

Installieren Sie zun\"achst die Distribution und anschließend den Editor. Beim ersten Start des Editors \"offnet sich ein 
Konfigurationsassistent, der zun\"achst nach dem Pfad der installierten Distribution fragt. 

Nach der Installation können k\"onnen Einstellungen z.~B.~f\"ur einen PostScript-Viewer gemacht werden. 
Dieser Schritt kann ohne Weiteres \"ubersprungen werden. Entscheidend sind die Einstellungen f\"ur den pdf-Viewer. 

Jetzt kann \LaTeX~verwendet werden. Um die Ausgabe eines Dokumentes zu erzeugen, muss das Dokument kompiliert werden (Ausgabe >
Aktives Dokument > Erstellen und betrachten).

\subsection{Beispiel Unterabschnitt: Aufbau eines \LaTeX-Dokuments}

Ein \LaTeX-Dokument besteht in der Regel aus folgenden Komponenten:
\begin{itemize}
	\item Pr\"aambel
	\item Titelseite
	\item Textteil
\end{itemize}

\subsection{Beispiel Unterabschnitt auf zweiter Ebene: Pr\"aambel}
In der Pr\"aambel werden global die Einstellungen f\"ur das gesamte Dokument definiert. Hierbei k\"onnen z.~B.~die Seitenr\"ander, 
der Zeilenabstand oder auch die Sprache f\"ur die Silbentrennung festgelegt werden. In der ersten Zeile eines jeden Dokumentes wird dabei
immer die zu verwendende Klasse festgelegt. Standardm\"aßig kann hier die Artikel-Klasse gew\"ahlt werden:

\texttt{\textbackslash documentclass[12pt,titlepage]\{article\}}

In den eckigen Klammern wird dabei u.a. die Standardschriftgr\"o\ss e f\"ur das gesamte Dokument festgelegt. 

Au\ss erdem werden in der Pr\"aambel die f\"ur das Dokument ben\"otigten Pakete festgelegt. Gebr\"auchlich sind vor allem folgende Pakete:
{\texttt{
\begin{itemize}
	\item \textbackslash usepackage[ngerman]\{babel\}
	\item \textbackslash usepackage[latin1]\{inputenc\}
	\item \textbackslash usepackage\{color\}
	\item \textbackslash usepackage[a4paper]\{geometry\}
	\item \textbackslash usepackage\{amssymb\}
	\item \textbackslash usepackage\{amsthm\}
	\item \textbackslash usepackage\{graphicx\}
\end{itemize}
}

Im vorliegenden Fall werden die Pakete in der Konfigurationsdatei \texttt{config.tex} festgelegt, deren Inhalt durch 
\texttt{\textbackslash input\{config\}} in das Hauptdokument \texttt{master.tex} inkludiert wird.

\subsubsection{Beispiel Unterabschnitt auf zweiter Ebene: Titelseite}

Nachdem die Dokumenten-Klasse und die zu verwendenden Pakete festgelegt worden sind,
folgt die Titelseite. Da die Titelseite bereits Teil des eigentlichen Dokuments ist, muss ihr
unbedingt der Befehl \texttt{\textbackslash begin\{document\}} vorausgehen. Am Ende des Dokuments sollte der Befehl
\texttt{\textbackslash end\{document\}} gesetzt werden. Alles was nach diesem Befehl steht, wird vom Compiler nicht mehr beachtet.

\section{Noch ein Beispiel-Abschnitt}

Der Textteil beinhaltet nun den eigentlichen Text des Dokuments.


% mehrere Grundlagen- und Forschungs-Kapitel
% !TEX root = master.tex
\chapter{Motivation}
\label{chapter:1}
Die Faszination für den genetischen Fortschritt und die Anpassungsfähigkeit in der Natur bildet die Grundlage für dieses Projekt. In der biologischen Welt lösen Lebewesen durch genetische Evolution und Mutationen Aufgaben zunehmend effizienter. Diese Beobachtung weckt das Interesse an der Frage, inwieweit ähnliche Prinzipien in Form von unsupervised learning auf den Bereich der Computerwissenschaften übertragen werden könnten. Speziell die Implementierung von genetikbasiertem Lernen in Computer-Agenten, angelehnt an natürliche Vorbilder, stellt ein faszinierendes Forschungsfeld dar.

Die Natur bietet unzählige Beispiele für effiziente Anpassungs- und Lernprozesse, die durch genetische Variationen und natürliche Selektion getrieben werden. Diese Prozesse haben zu einer beeindruckenden Vielfalt und Spezialisierung der Arten geführt. In der Informatik könnten ähnliche Mechanismen genutzt werden, um lernfähige Systeme zu entwickeln, die sich selbstständig an neue Aufgaben und Umgebungen anpassen können. Die Herausforderung liegt darin, Konzepte der natürlichen Evolution in Algorithmen zu übersetzen, die in der Lage sind, komplexe Probleme effektiv zu lösen.

Der Ansatz, genetikbasierte Lernmethoden auf Computer-Agenten anzuwenden, bietet das Potenzial, die Grenzen herkömmlicher KI-Methoden zu überwinden. Insbesondere könnte dieser Ansatz dazu beitragen, das Problem der Überanpassung (Overfitting) zu verringern, da genetische Algorithmen eine natürliche Tendenz zur Exploration und Diversifikation aufweisen. Zudem ermöglicht die Verwendung von unsupervised learning Methoden eine flexiblere Anpassung an unbekannte oder sich verändernde Umgebungen, was in vielen realen Anwendungsfällen von großer Bedeutung ist.

Diese Arbeit zielt darauf ab, die Machbarkeit und Effizienz von genetikbasiertem Lernen in Computer-Agenten zu untersuchen. Durch die Kombination von Theorien aus der Biologie und Informatik wird versucht, ein tieferes Verständnis dafür zu entwickeln, wie maschinelles Lernen durch die Prinzipien der Evolution bereichert werden kann. Die Ergebnisse könnten weitreichende Implikationen für die Entwicklung zukünftiger KI-Systeme haben und einen bedeutenden Schritt in Richtung der Schaffung autonomer, adaptiver Lernsysteme darstellen.





\chapter{Theoretische Grundlagen}
\label{chapter:2}

\section{Grundlagen von NEAT}
(2 Seiten)
---- Historisches Wachsen von genetischen Algos - erste ML powered Chatbots, Neuronale Netze, Supervised Learning, Big Data und am Ende das unsupervised deep-learning mit sich selbst adjustierenden Gewichte. Darauf basierend im nächsten Schritt modellierung natürlichen Verhaltens durch genetik. 
---- Bestandteile, N=Neuro-NeuronaleNetze, E=Evolution-Mutation, Crossover, Selection, A=Augmenting- kleinstes NN am Anfang-falls zu schlecht mehr komplexität, T=Topology-Anzahl an Neuronen und Layer sowie Aufbau dessen. Phenotype (ein neuronales Netz), Genotype (ein gewisses Genset), Speciation. Ein Genom mit mehreren Genen, entweder Knoten, oder Kanten-gen. Kantengene haben Innovationsnummern-beschreibt wann eine Kante kreiert wurde, diese werden über alle Netzwerke hinweg gleich sein-wenn ein Netzwerk eine neue Kante erstellt die keine vorheriges hatte bekommt das eine hohe Innovationsnummer. Falls die woanders schon existiert wird die Nummer von der Kante reduziert-Grundlage für gutes Crossover-das passiert nur auf Kantengene, da die Knotengene von dieses abgeleitet werden können-eine Knotenkanten klasse besitzt typischerweise fünf Attribute: 1. Innovationsnummer, 2. Ursprungsknoten, 3. Endknoten, 4. Gewicht, 5. IstAktiv. Ein Knoten hat zwei Attribute: 1. Knotennummer und 2. Knotentyp (Sensor, Hidden, Output) Unterscheidung von excess (am Ende, werden nur vererbt wenn das Elternteil höhere Scores trifft) und disjoint gene (werden immer vererbt und gemerged). DAS WARS ZU CROSSOVER. Fünf Mutations typen, entweder eine Verbindung mutieren das heißt eine neue Verbindung mit dem Gewicht zwischen [-2;2], zweitens man fügt einen Knoten irgendwo dazu auf eine zufällige Verbindung, vorherige Verbidnung behält ihren Wert, die neue Verbindung bekommt Gewicht 1. Drittens eine Verbindung aktivieren, oder deaktivieren. Viertens die Gewichtsverschiebung, multipliziert ein zufälliges Gewicht mit einer Nummer zwischen [0;2]. Fünftens die Gewichtsneusetzung, komplett neues Gewicht zwischen [-2;2]. ENDE MUTATION. Selektion: Unterschiedliche Genome und jedes davon hat einen bestimmten Score in der Simulation erreicht. Speziefizierung  der Genome-Wenn die Distanz zweier Genome unterhalb einer gewissen Grenze ist, gruppiert man diese in eine Spezies - geht nicht um den Score sondern um die Gewichte, wenige Excess und Disjoint Gene. Falls die Distanz zu groß ist und es keine Spezies dafür gibt, erstellt man eine neue Spezies startend mit diesem Genom. Sortieren der Genome innerhalb der einzelnen Spezien basierend auf ihrem Score, absteigend. Dann selektiert man basierend auf einem Schwellenwert eine gewisse Anzahl an Genomen aus jeder Spezie (häufig 50\%) und vernachlässigt alle restlichen im weiteren Programmablauf. Auch sehr schlechte Genome können diesen Prozess durchstehen, da Sie in einer kleinen Spezie sind. Das garantiert Innovation auf der Topologie Ebene, welches durch kleiner Mutationen vielleicht noch wesentlich besser werden könnte. (Grafik zum Grundlegenden Konzept von NEAT)

\section{Grundlagen von simplen Computerspielen}
(1 Seiten)
---- Historische Bedeutung von Computerspielen (hohe gesellschaftliche Durchdringung von Spielen egal ob Militär oder Familie, Kowmpetitv aber begrenzt, zuerst Brettspiele dann Computer, soziales Tun, Vorfornt in der Analyse von menschlichen psychologischen Verhalten)
---- Bestandteile (variierende Umgebung, neue Input Reize, ein Ziel, z.B. Highscore oder Überleben, häufig Kompetitv, Spieler steuert einen Agenten in einer simplifizierten Welt, visuell häufig ausgeklügelt aber nicht ausschlaggebend für gutes Spiel -> Überleitung zu den Grundlagen der Visualisierung)

\section{Grundlagen zum Thema Visualisierung}
(2 Seiten)
---- Ganz Basic warum Visualisierung -> weil leicht konsumierbar von Menschen, häufig leichter verständlicher und kompensierter als Text. Vergleiche Dateispeichergröße zwischen Text, Audio und Video 

https://open.oregonstate.education/aandp/chapter/13-1-sensory-receptors/
https://www.frontiersin.org/articles/10.3389/fncir.2015.00051/full  [Add to Citavi project by DOI] 
https://www.ncbi.nlm.nih.gov/pmc/articles/PMC6091269/  [Add to Citavi project by PMC ID] 
https://publishup.uni-potsdam.de/opus4-ubp/frontdoor/deliver/index/docId/5047/file/digarec06\_S116\_156.pdf


---- Psychologische Effekte von Visualisierung (vgl. Text/Bild/Videos) in spielen, dann verallgemeinert auch beim lernen, leichteres lernen? schnelleres lernen? (Grafik zum biologischen sehen) 
---- Neuster Stand bei visuellen Effekten und Erkenntnisse, Mikroanimationen, Zerlegung komplexer Themen in Videos - Bewegtbildproduktionen. Trend zu immer kürzeren Videos und schnellere Schnittfolge um Aufmerksamkeit zu bewahren. Finder Anklang auch in der Wissenschaft, leichtere Vermittlung komplexer Inhalte an viele Zuhörer - mehr Wissensvideos zum Beispiel auch im Informatik/Mathematik und Biologie Bereich. 

"Eine ausreichende Lernzeit scheint sich
demzufolge verstehensförderlich auszuwirken." (S. 122 d-nb.info)
Visualisierungen erhöhen den Lernerfolg und reduzieren die mentale Belastungen bei dem Verständnis von wissenschaftlichen Texten. (vgl. S. 118 d-nb.info)
Zudem reduzieren Videos die mentale Belastung im Vergleich zu statischen Bildern beim Verständnis neuer Konzepte. (vgl. S. 324 hbz-nrw.de)



% !TEX root = master.tex
\chapter{Aufwandsschätzung}
%Zielsetzung: Planning Poker durchführen zur Aufwandsschätzung für die in
%Szenario 2 definierten Arbeitspakete
%– Bitte verwendet hierzu die Datei 2_Aufwandsplanung
%– Planning Poker gibt es verschiedentlich online
%– Wählt eine Referenz-Story (Arbeitspaket) aus und begründet eure Wahl
%– Beschreibt eure Erfahrungen mit Planning Poker: Was hat gut funktioniert, wo sind
%Schwierigkeiten aufgetreten?
%– Wie wurde mit Meinungsverschiedenheiten umgegangen?
\label{chapter:3}
Das Planning Poker wurde im Anschluss an die Fertigstellung der Arbeitspakete durchgeführt, wobei jede Runde dokumentiert und in Tabelle \ref{tab:Planning Poker} festgehalten wurde. Die finale Aufwandsschätzung (Beschluss) entspricht stets dem Mittelwert aller Einzelschätzungen und wurde berechnet, sobald die kleinste und größte Einschätzung maximal eine Karte auseinanderlagen oder die dritte Iteration einer Schätzrunde erreicht wurde. 

\begin{table}[!h]
	\footnotesize
	\centering
	\renewcommand{\arraystretch}{1.3}
	\begin{tabularx}{\linewidth}{|X|l|c|c|c|c|r|}
		\hline
		\textbf{Arbeitspaket} & \textbf{Iterationen} & \textbf{Eric} & \textbf{Benedikt} & \textbf{David} & \textbf{Lasse} & \textbf{Beschluss} \\ \hline
		AP-K-1 & 1 & 3 & 5 & 13 & 8 & \\ 
		& 2 & 5 & 5 & 8 & 8 & \textbf{6,5} \\ \hline
		AP-K-2 & 1 & 8 & 8 & 13 & 40 & \\ 
		& 2 & 13 & 13 & 13 & 20 & \textbf{14,75} \\ \hline
		AP-E-1 & 1 & 5 & 8 & 13 & 100 & \\ 
		& 2 & 8 & 8 & 13 & 40 & \\ 
		& 3 & 8 & 8 & 13 & 20 & \textbf{12,25} \\ \hline
		AP-E-2 & 1 & 8 & 13 & 13 & 40 & \\ 
		& 2 & 13 & 13 & 13 & 13 & \textbf{13} \\ \hline
		AP-E-3 & 1 & 20 & 20 & 20 & 20 & \textbf{20} \\ \hline
		AP-E-4 & 1 & 20 & 20 & 20 & 40 & \textbf{25} \\ \hline
		AP-T-1 & 1 & 5 & 8 & 8 & 40 & \\ 
		& 2 & 5 & 5 & 8 & 13 & \\ 
		& 3 & 5 & 5 & 8 & 8 & \textbf{6,5} \\ \hline
		AP-T-2 & 1 & 8 & 8 & 13 & 13 & \textbf{10,}5 \\ \hline
		AP-P-1 & 1 & 40 & 40 & 40 & 40 & \textbf{40} \\ \hline
		AP-P-2 & 2 & 8 & 8 & 8 & 13 & \textbf{9,25} \\ \hline
		AP-R-1 & 1 & 8 & 8 & 13 & 40 & \\ 
		& 2 & 8 & 8 & 8 & 8 & \textbf{8} \\ \hline
		AP-R-2 & 1 & ? & 13 & 13 & 40 & \\ 
		& 2 & 13 & 20 & 40 & 40 & \\ 
		& 3 & 13 & 20 & 20 & 20 & \textbf{18,25} \\ \hline
	\end{tabularx}
	\caption{Planning Poker}
	\label{tab:Planning Poker}
	\normalsize
\end{table}

\section{Positive Erfahrungen}

Insgesamt war unserer Erfahrung mit der Methode Planning Poker positiv. Es wurde sachlich aber engagiert diskutiert und letztlich konnte allen Arbeitspaketen eine Aufwandsschätzung zugeordnet werden. Die Verwendung des Smartphone für die Abstimmung sowie das geheime Abstimmen mit anschließender Aufdeckung der Ergebnisse, gibt der Methode einen spielerischen Ansatz, wodurch sie mehr Spaß gemacht hat, als eine offene Diskussion der Aufwände ohne Hilfsmittel. Es wurden alle Regeln eingehalten und alle Mitglieder waren dazu bereit, Kompromisse einzugehen und sich in ihren Einschätzung einander anzunähern. Der gesamte Prozess hat etwa 90 Minuten in Anspruch genommen und wurde von allen Teilnehmern als interessant und produktiv wahrgenommen.

\section{Negative Erfahrungen}

Negativ aufgefallen ist uns, dass man dazu tendiert, eine der mittleren Ausprägungen zu wählen und Extreme zu meiden. Diese Tendenz hat im Laufe der Runden weiter zugenommen. So wurden zu Beginn mitunter auch sehr kleine (3) und große (100) Karten gewählt, während am Ende meist die mittleren Karten (8, 13, 20) gewählt wurden. 
Zudem haben wir festgestellt, dass man dazu neigt, vergangene Einschätzungen als Richtwert zu nehmen. So wurde oftmals argumentiert, dass ein Arbeitspaket eine gewissen Umfang haben muss, damit es in einem sinnvollen Verhältnis zur Gesamtprojektlaufzeit steht. Dies ist jedoch problematisch, sobald der Richtwert selbst falsch eingeschätzt wurde. Ist die Gesamtprojektlaufzeit beispielsweise zu hoch angesetzt, dann würden die Arbeitspakete künstlich in die Länge gezogen werden, was zu unrealistischen Aufwandsschätzungen und einer unnötigen Verzögerung des Projektfortschritts führt. 
Außerdem lässt sich kritisieren, dass das Argumentationsgeschick der Teilnehmer im Zweifel ausschlaggebender ist als die objektiven Fakten. So kann eine diskussionsfreudige Person die anderen Teilnehmer mithilfe rhetorischen Geschicks von seinem Standpunkt überzeugen, selbst wenn seine Schätzung unrealistisch ist.
Ebenso problematisch ist die Tatsache, dass keine Diskussionsrunde eröffnet wird, wenn die Einschätzungen der Teilnehmer übereinstimmen. So ist es durchaus möglich, dass die Schätzungen der Projektmitglieder verschiedene Aspekte berücksichtigen, womit ein Austausch durchaus sinnvoll wäre. Denn nur so ergibt sich ein umfassendes Gesamtbild aller Teilaspekte und eine akkurate Aufwandsschätzung kann erstellt werden.
Des Weiteren gab es einige Arbeitspakete bei denen eine Schätzung des Aufwands nur schwer möglich war. Dies betraf oft Aufgaben, deren Umfang durch vorangegangene Tätigkeiten bestimmt wurden. Wie viel Zeit für das Beheben von Fehlern während der Pilotierung nötig ist, hängt beispielsweise sehr stark von der Qualität des vorangegangenen Testings ab. Wurden dort alle Edge-Cases berücksichtigt und abgedeckt, werden später auch weniger Fehler auftreten.
Zuletzt lässt sich auch die generelle Effizienz des Verfahrens in Frage stellen. Für die Bewertung aller 12 Arbeitspakete haben wir etwa 90 Minuten gebraucht. Bei 4 Teilnehmern sind dies bereits 6 Stunden Arbeitszeit, welche in die Umsetzung der Anforderungen hätten investiert werden können. Angesichts der Tatsache, dass in der zweiten und dritten Iteration einer Schätzrunde nur selten neue Argumente vorgetragen werden, lässt sich zudem kritisch hinterfragen, ob es zwingend drei Iterationen geben muss.

\section{Umgang mit Meinungsverschiedenheiten}

Wie der aus Tabelle \ref{tab:Planning Poker} hervorgeht, gingen die Einschätzungen der Teilnehmer teils stark auseinander. So hat Eric dem Arbeitspaket AP-E-1 in der ersten Iteration 5 Story Points zugeordnet, während es bei Lasse 100 waren. Diese große Differenz ist zum einen natürlich auf die subjektiven Erfahrungen und Kenntnisse der einzelnen Teilnehmer, teilweise aber auch auf Verständnisprobleme oder unterschiedliche Interpretationen der Arbeitspakete zurückzuführen. Insofern zweiteres der Fall war, wurde sich auf eine Interpretation der Aufgabe geeinigt und mit der nächsten Iteration fortgefahren. Im Falle subjektiver Meinungsunterschiede wurde allen Teilnehmern die Möglichkeit geboten ihren Standpunkt zu begründen und anschließend konnte frei diskutiert werden. Sobald keine neuen Argumente mehr vorgetragen wurden, wurde die Diskussion abgebrochen und eine neue Iteration gestartet. Mit Ausnahme von Arbeitspaket AP-E-1 konnte durch dieses Vorgehen stets ein Kompromiss erzielt werden. Prinzipiell waren alle Teilnehmer bereit von ihrem Standpunkt abzurücken, es kam allerdings sehr selten vor, dass jemand bereit war in einer Iteration mehr als eine Zahl der Fibonacci-Folge auf die anderen zuzukommen.
% !TEX root = master.tex
\chapter{Ressourcenplanung}
%© DHBW | Svenja Reimann
%Wir brauchen ein Projektteam!
%– Stellt ein Scrum-Team zusammen aus den verfügbaren Ressourcen aus der Datei
%3_Ressourcenplanung
%– Ein Scrum Team hat zwischen 3 & 9 Personen – schätzt ab, wie groß das Team
%realistischerweise sein sollte zur Erfüllung des Projekt-Scopes in der von euch im
%Projektsteckbrief definierten Projektlaufzeit
%– Zur Vereinfachung: der Projektleiter (euch stehen in der Datei zwei zur Auswahl) übernimmt
%die Rolle des Product Owners, um einen agile Master müsst ihr euch nicht kümmern!
%– Begründet eure Wahl der einzelnen Projektmitglieder
%– Bestimmt die Gesamtkosten eurer Ressourcenplanung anhand der genannten Tagessätze
%und eurer Annahmen zur Aufwandsschätzung der Arbeitspakete

%- Projektteam zusammen stellen
%- Brauchen ein Scrum Team ohne agilen Master
%- Gesamtkosten der Ressourcenplannung aufgrund von Tagessätzen (Wochenenden, Feiertage und %Urlaub rausrechnen oder so)
%- Es gibt 8 interne und 7 externe
%- basierend auf dem Projekt ist der schnelle und reibungslose Erfolg am wichtigsten also %lieber etwas overstaffed als unter staffed (>6 peeps)!
%- Nicht so Kostendruck weil strategisch sehr wichtiges Projekt -> Muss langfristig %funktionieren -> Interne Verantwortlichkeit wichtig! 
%- Entwicklung kann outsourced werden, PM verantwortlichkeit sollte aufgrund von IP und  %governance Gründen im Unternehmen bleiben
%- Diverses und Kommunikatives Team wichtig!
%- Diversität ist wichtig! Geschlecht, Erfahrung und Work ethos
%- Kein only Experten Team -> billigere Tagessätze!
%- Backend implementierung wichtig frontend nicht allzu wichtig
%- basierend auf dem Projekt ist viel Testing wichtig!
%- basierend auf dem Projekt ist ein Frontend nicht wirklich wichtig. 
\label{chapter:4}
\section{Ausgangssituation}
\textbf{Entscheidungsrahmen}
\vspace*{0.1cm}

Die Zusammenstellung eines Projektteams für ein strategisch wichtiges IT-Projekt kann durch die Anwendung eines fundierten Entscheidungsrahmens optimiert werden. Dieser ermöglicht die Berücksichtigung und Abwägung aller relevanten Aspekte, um ein Scrum-Team zusammenzustellen.

Der Erfolg des Projekts bildet einen zentralen Faktor im Entscheidungsrahmen, wobei ein besonderer Fokus auf einem schnellen und reibungslosen Projektablauf liegt. Dabei profitiert das Projekt von einem eher überbesetzten Zustand als Risikovorbeugung gegen eine potenzielle Unterbesetzung.

Der Entscheidungsrahmen bezieht auch langfristige Projektstabilität ein. Dabei wird die strategische Bedeutung des Projekts und die Notwendigkeit einer starken internen Verantwortung erkannt. Das Behalten der Projektmanagement-Verantwortung intern erfüllt Intellectual Property- und Governance-Anforderungen, während die Entwicklung als möglicherweise outsourcebar betrachtet wird.

Vielfalt und Kommunikationsfähigkeit sind weitere Schlüsselelemente des Entscheidungsrahmens bei der Teamzusammensetzung. Angestrebt wird eine Diversität hinsichtlich Geschlecht, Erfahrung und Arbeitsmoral. Dies bringt den Vorteil eines ausgewogenen Mixes aus Expertenwissen und kosteneffizienten Ressourcen mit sich. Zudem fließen dadurch diverse Meinungen in das Projekt mit ein und ein Verlieren in bestimmte Spezifika wird unwahrscheinlicher.

Die Kostenplanung ist ein integraler Bestandteil des Entscheidungsrahmens. Das Ziel ist es, trotz der strategischen Bedeutung des Projekts und dem fehlenden akuten Kostendruck eine langfristige Rentabilität sicherzustellen. Dabei wird versucht, sowohl interne als auch externe Ressourcen optimal zu nutzen.
\pagebreak
\vspace*{0.5cm}

\textbf{Entscheidungskriterien}
\vspace*{0.1cm}

Im Entscheidungsrahmen für die Ressourcenplanung werden vier zentrale Aspekte berücksichtigt und auf einer Skala von \enquote{+ +} (sehr positiv) bis \enquote{- -} (sehr negativ) für jedes potenzielle Mitglied innerhalb einer benötigten Rolle bewertet. Die finale Bewertung ergibt sich durch eine Verrechnung dieser Einzelbewertungen.

\begin{itemize}
\item \textbf{Qualität:} Hierbei werden die technische Fähigkeit und die Rolle des Kandidaten im Team berücksichtigt. Die Erfahrung des Kandidaten spielt dabei eine entscheidende Rolle.
\item \textbf{Arbeitsmoral:} Dieser Aspekt betrachtet, wie gut der Kandidat ins Team passt und welchen strategischen Wert er für das Projekt hat. Darüber hinaus wird davon ausgegangen, dass jüngere Mitarbeiter aufgrund von Aufstiegsambitionen oft motivierter sind.
\item \textbf{Diversität:} Diversität in einem Team kann eine breite Palette von Perspektiven und Lösungen bieten, wodurch die Problemlösungskompetenz verbessert wird. Hierbei geht es sowohl um Geschlecht, Erfahrung und interne oder externe Zugehörigkeit.
\item \textbf{Kosten:} Die Kosten sollten sich im Rahmen bewegen. Aus Projektsicht sind niedrige Kosten von Vorteil, da dadurch häufiger potenzielle Iterationen durchgeführt werden können.
\end{itemize}

Dieser Entscheidungsrahmen bietet den Vorteil einer systematischen und objektiven Bewertung von Kandidaten, was zu einer effektiveren, effizienteren und nachvollziehbaren Ressourcenplanung führt.

\vspace*{0.2cm}

\textbf{Scrum Team Übersicht}

In Folgenden werden alle möglichen Kandidaten, sowie deren Spezifika, für dieses Projekt aufgelistet. Diese bilden die Grundlage für die Evaluierung und Staffing des Teams. \\

\pagebreak
\thispagestyle{empty}
\begin{center}
\small
\begin{tabularx}{\textwidth}{|>{\arraybackslash}p{2.2cm}|X|>{\arraybackslash}p{.9cm}|X|>{\arraybackslash}p{1.6cm}|>{\arraybackslash}p{2.1cm}|>{\arraybackslash}p{1.6cm}|}
\hline
\textbf{Name} & \textbf{Rolle} & \textbf{Typ} & \textbf{Erfahrung} & \textbf{Verfügbar} & \textbf{Besonders} & \textbf{Kosten} \\
\hline
Anna Schmidt & IT Architect & Intern & 4 Jahre & max. 50\% & Kommuni-kationsstark & 550€/Tag \\
\hline
Mira Bellenbaum & Marketing-Kauffrau & Intern & Neu & 100\% & Egozen-trisch & 520€/Tag \\
\hline
Martin Müller & Leiter IT Abteilung & Intern & 15 Jahre & 0\%  & Kontroll-bedürfnis & 720€/Tag \\
\hline
Harry Mayer & Entwickler & Intern & 1 Jahr & 100\% & Disruptive Methoden & 600€/Tag \\
\hline
Stefan Schmitt & Entwickler & Intern & 3 Jahre & 100\% & Teilzeit-wunsch Frontend Erfahrung & 600€/Tag \\
\hline
Tom Schulze & Entwickler & Intern & 4 Jahre & 100\% & SAP Backend Developer & 600€/Tag \\
\hline
Susi Sonnenschein & Senior Business Analyst & Intern & 15 Jahre &  50\% & Fachwissen & 640€/Tag \\
\hline
Franz Urlaub & Projektlei-ter & Intern & 5 Jahre & 100\% & Laissez-faire Führungsstil & 830€/Tag \\
\hline
Max Mustermann & Senior IT Project Manager & Extern & 8 Jahre & Freelancer & Golfspieler & 1200€/Tag \\
\hline
Maria Musterfrau & Marketing-Kauffrau & Extern & Viel Erfahrung & Stunden-basis & Egozen-trisch & 100€/Stun-de \\
\hline
Mace Windu & Business Analyst & Extern & 11 Jahre & Freelancer & Reisebereit-schaft & 970€/Tag \\
\hline
Sam Gamdschie & Frontend Developer & Extern & 6 Jahre & 80\% & Handwerkli-che Hobbies & 870€/Tag \\
\hline
Rubeus Hagrid & Backend Developer & Extern & 2 Jahre & 80\% & Parallel-studium & 790€/Tag \\
\hline
Tim Tiek & Test Manager & Extern & 10 Jahre & Zertifiziert& Viel Erfahrung mit HP Testtools & 860€/Tag \\
\hline
Tom Tonk & Software Tester & Extern & Einige Jahre & Zertifiziert& Internatio-nale Projekte & 760€/Tag \\
\hline
\end{tabularx}
\end{center}

\vspace*{0.5cm}

\textbf{Scrum Team Anforderungen}
\vspace*{0.1cm}

Scrum ist ein agiles Framework, das auf iterativen und inkrementellen Prozessen basiert. Es betont die Zusammenarbeit und Anpassungsfähigkeit in Teams zur effektiven Lösung komplexer Probleme. 

\begin{figure}[h]
	\centering
	\includegraphics[scale=1.3]{img/Scrum.pdf}
	\captionsetup{format=hang}
	\caption{\label{fig:Scrum}Darstellung des Scrum Team Konzept}
\end{figure}
%-Scrum DEFINITION MIT Überlappenden Kreisen

In dieser Grafik ist ein Scrum-Team abgebildet, das aus dem \textit{Scrum Master}, dem \textit{Product Owner} und dem \textit{Entwicklungsteam} besteht. Der \textit{Scrum Master} gewährleistet die Einhaltung der Scrum-Prinzipien, während der \textit{Product Owner} die Produktvision und die Priorisierung der Anforderungen steuert. Das \textit{Entwicklungsteam} ist verantwortlich für die Entwicklung und Auslieferung des Produkts. Bei effektiver Zusammenarbeit dieser drei Rollen entsteht eine \textit{optimale Zusammenarbeit}, die sich durch Flexibilität, Produktivität und Kreativität auszeichnet. Schließlich muss das fertige Produkt in diesem Fall noch erfolgreich durch eine passende \textit{Vermarktung} verkauft werden.

In diesem speziellen Projekt soll das Scrum-Team aus mindestens sechs Personen bestehen, um eine Unterbesetzung zu vermeiden und einen hohen Output zu gewährleisten. Generell lässt sich sagen, dass für die Rolle des Product Owner, der als Projektmanager agiert, alle Projektleiter und Abteilungsleiter in Betracht gezogen werden können, da sie Management- und Teamleitungsexpertise besitzen. Währenddessen besteht die Rolle des Scrum Masters darin, das Entwicklungsteam zu unterstützen und zu kontrollieren. Häufig ist der Scrum Master auch in gewissem Maße bei der Entwicklung beteiligt. Daher können Business Analysten oder Architekten als potenzielle Kandidaten für diese Rolle geeignet sein. Erstere sind darauf spezialisiert, analytische Fähigkeiten mit hoher Kommunikationskompetenz zu verbinden, was bei der Koordinierung von Aufgaben hilfreich ist. Letztere bieten aufgrund ihres technischen Wissens einen nahbaren Ansprechpartner für Entwickler und haben durch ihre Rolle ein umfassenderes Bild des Produkts und können es besser nach außen vertreten. Das Entwicklungsteam benötigt hauptsächlich Entwickler, die das Produkt funktional implementieren und liefern können. Das Frontend ist hierbei aufgrund des B2B-Marktes nicht der primäre Fokus und wird weniger intensiv betrachtet. Das Testing ist dafür umso relevanter und sollte von einem zertifizierten Tester mit großer Sorgfalt durchgeführt werden. Anhand mehreren kleineren Projekten soll die Vermarktung durch professionelle Kampagnen stattfinden. Daraus ergibt sich die Anzahl der Teammitglieder und ihre spezifischen Rollen:

\begin{itemize}
\item \textit{1x Product Owner (PO)}: Ein einzelner PO ist ideal, um eine klare und konsistente Produktvision zu gewährleisten und Priorisierungsentscheidungen zu treffen.

\item \textit{1x Scrum Master}: Ein Scrum Master sorgt dafür, dass die Scrum-Methodik effektiv umgesetzt wird, was für den Erfolg eines agilen Projekts entscheidend ist.

\item \textit{1x Marketing}: Eine dedizierte Marketing-Person ist erforderlich, um das Produkt effektiv zu vermarkten und den Kunden am Ball zu behalten.

\item \textit{$ \geq \ $3x Entwickler}: Ein größeres Entwicklerteam ermöglicht eine hohe Entwicklungsrate und sorgt für die nötige Qualität und den Wissensaustausch im Team.

\item \textit{1x Tester}: Ein dedizierter Tester ist essenziell, um die Qualität des Produkts zu gewährleisten und eine frühzeitige Erkennung und Behebung von Fehlern zu ermöglichen.
\end{itemize}

Diese siebenköpfige Zusammenstellung gewährleistet eine umfassende Abdeckung aller erforderlichen Fachkompetenzen für dieses Projekt.

\vspace*{0.5cm}

\textbf{Scrum Team Auswahl}
\vspace*{0.1cm}

\textbf{Product Owner:} Der Product Owner ist für die Definition und Priorisierung der Produktanforderungen in Form von User Stories im Produkt-Backlog zuständig. Er oder sie arbeitet eng mit dem Entwicklungsteam und den Stakeholdern zusammen, um sicherzustellen, dass die Vision und Ziele des Produkts klar verstanden werden. Hier stehen drei Kandidaten zur Auswahl.

\begin{center}
\small
\begin{tabularx}{\textwidth}{|>{\arraybackslash}p{2.2cm}|X|>{\arraybackslash}p{.9cm}|X|>{\arraybackslash}p{1.6cm}|>{\arraybackslash}p{2.1cm}|>{\arraybackslash}p{1.6cm}|}
\hline
\textbf{Name} & \textbf{Rolle} & \textbf{Typ} & \textbf{Erfahrung} & \textbf{Verfügbar} & \textbf{Besonders} & \textbf{Kosten} \\
\hline
Martin Müller & Leiter IT Abteilung & Intern & 15 Jahre & 0\% & Kontroll-bedürfnis & 720€/Tag \\
\hline
Franz Urlaub & Projektlei-ter & Intern & 5 Jahre & 100\% & Laissez-faire Führungsstil & 830€/Tag \\
\hline
Max Mustermann & Senior IT Project Manager & Extern & 8 Jahre & Freelancer & Golfspieler & 1200€/Tag \\
\hline
\end{tabularx}
\end{center}


\begin{enumerate}
\item \textbf{Qualität}:
Kandidat eins zeigt das größte Potenzial aufgrund langer Arbeitserfahrung, ist jedoch komplett ausgelastet. Er wird trotzdem berücksichtigt unter der Annahme, dass dieser wenn er der klare best-fit ist, andere Projekte zugunsten diesem abgegeben kann. Danach folgt Kandidat drei, und am Ende Kandidat zwei mit nur fünf Jahren Projektmanagementerfahrung.

\item \textbf{Arbeitsmoral}:
Kandidat zwei und drei teilen sich den ersten Platz. Kandidat zwei ist jünger, hat jedoch einen "laissez-faire" Führungsstil. Dafür hat Kandidat drei vermutlich aufgrund seines sportlichen Hintergrunds kompetitive Ambitionen und kompensiert damit sein etwas höhere Seniorität. Kandidat eins ist eigentlich komplett ausgelastet und hat vermutlich weniger Lust noch ein Zusatzprojekt in Überstunden zu leiten.

\item \textbf{Diversität}:
Kandidat zwei punktet leicht in Bezug auf Diversität, da ein offener Führungsstil auch allgemein mehr Offenheit vermuten lässt.

\item \textbf{Kosten}:
Günstiger ist besser.
\end{enumerate}


\begin{center}
\small
\begin{tabularx}{\textwidth}{|l|X|X|X|X|X|X|}
\hline
\textbf{Name} & \textbf{Qualität} & \textbf{Arbeits-moral} & \textbf{Diversität} & \textbf{Kosten} & \textbf{Ergebnis} \\
\hline
Martin Müller & ++ & - - & - & ++ & (+) \\
\hline
Franz Urlaub & - & ++ & + & + & (+++) \\
\hline
Max Mustermann & + & ++ & - & - & (+) \\
\hline
\end{tabularx}
\end{center}

\textit{\textbf{- Franz Urlaub bietet sich am besten als Product Owner an.}}
\vspace*{0.1cm} \\
\textbf{Scrum Master:} Der Scrum Master ist dafür verantwortlich, sicherzustellen, dass das Team die Scrum-Prinzipien und -Praktiken einhält. Er oder sie entfernt Hindernisse, die die Arbeit des Entwicklungsteams behindern könnten, und hilft dem Team dabei, sich selbst zu organisieren und hochwertige Produkte zu liefern. Darauf basierend bieten sich ebenfalls drei Kandidaten an.

\begin{center}
\small
\begin{tabularx}{\textwidth}{|>{\arraybackslash}p{2.2cm}|X|>{\arraybackslash}p{.9cm}|X|>{\arraybackslash}p{1.6cm}|>{\arraybackslash}p{2.1cm}|>{\arraybackslash}p{1.6cm}|}
\hline
\textbf{Name} & \textbf{Rolle} & \textbf{Typ} & \textbf{Erfahrung} & \textbf{Verfügbar} & \textbf{Besonders} & \textbf{Kosten} \\
\hline
Anna Schmidt & IT Architect & Intern & 4 Jahre & max. 50\% & Kommuni-kationsstark & 550€/Tag \\
\hline
Susi Sonnenschein & Senior Business Analyst & Intern & 15 Jahre & max. 50\% & Fachwissen & 640€/Tag \\
\hline
Mace Windu & Business Analyst & Extern & 11 Jahre & Freelancer & Reise-bereitschaft & 970€/Tag \\
\hline
\end{tabularx}
\end{center}

\begin{enumerate}
\item \textbf{Qualität}:
Kandidatin zwei zeigt das größte Potenzial aufgrund langjähriger Arbeitserfahrung. Danach folgt Kandidat drei mit elf Jahren Erfahrung, und am Ende Kandidatin eins mit nur vier Jahren Projektmanagementerfahrung.

\item \textbf{Arbeitsmoral}:
Kandidatin eins - Höchster Arbeitswille aufgrund ihres Engagements. Kandidat drei zeigt durch seine Reisebereitschaft eine höhere Arbeitsmoral als Kandidatin zwei.

\item \textbf{Diversität}:
Kandidatin eins und zwei punkten aufgrund ihres Geschlechts. Kandidat zwei und Kandidat drei punkten mit ihrer Seniorität, da diese so in bisherige Rollen (PO) nicht abgedeckt ist.

\item \textbf{Kosten}:
Bei den Kosten gilt dasselbe wie zuvor.
    
\end{enumerate}


\begin{center}
\small
\begin{tabularx}{\textwidth}{|l|X|X|X|X|X|X|}
\hline
\textbf{Name} & \textbf{Qualität} & \textbf{Arbeits-moral} & \textbf{Diversität} & \textbf{Kosten} & \textbf{Ergebnis} \\
\hline
Anna Schmidt & - - & ++ & + & ++ & (+++) \\
\hline
Susi Sonnenschein & ++ & - & ++ & + &(++++) \\
\hline
Mace Windu & + & + & ++ & - & (+++) \\
\hline
\end{tabularx}
\end{center}

Basierend auf der Tabelle eignet sich Susi Sonnenschein am besten. Allerdings ist sie nur zu 50\% verfügbar. Daher teilt sie sich diese Rolle mit Kandidatin eins, die ebenfalls zu 50\% verfügbar ist.

\textit{\textbf{- Susi Sonnenschein und Anna Schmidt bieten sich am besten als Duo-Scrum Master an.}}
\vspace*{0.1cm} \\
\textbf{Entwicklungsteam:} Das Entwicklungsteam ist für die Entwicklung und Lieferung des Produkts verantwortlich. Es arbeitet selbstorganisiert und crossfunktional, um im Rahmen von Sprints inkrementelle Produktversionen zu liefern. Das Team ist verantwortlich für das Design, die Codierung, das Testen und die Qualitätssicherung des Produkts. Es werden mindestens zwei Entwickler, ein Teilzeit-Frontend-Entwickler und ein Tester gesucht.
\begin{center}
\small
\begin{tabularx}{\textwidth}{|>{\arraybackslash}p{2.2cm}|X|>{\arraybackslash}p{.9cm}|X|>{\arraybackslash}p{1.6cm}|>{\arraybackslash}p{2.1cm}|>{\arraybackslash}p{1.6cm}|}
\hline
\textbf{Name} & \textbf{Rolle} & \textbf{Typ} & \textbf{Erfahrung} & \textbf{Verfügbar} & \textbf{Besonders} & \textbf{Kosten} \\
\hline
Harry Mayer & Entwickler & Intern & 1 Jahr & 100\% & Disruptive Methoden & 600€/Tag \\
\hline
Stefan Schmitt & Entwickler & Intern & 3 Jahre & 100\% & Frontend Wissen & 600€/Tag \\
\hline
Tom Schulze & Entwickler & Intern & 4 Jahre & 100\% & SAP Backend Developer & 600€/Tag \\
\hline
Sam Gamdschie & Frontend Developer & Extern & 6 Jahre & 80\% & Handwerkli-che Hobbies & 870€/Tag \\
\hline
Rubeus Hagrid & Backend Developer & Extern & 2 Jahre & 80\% & Parallelstu-dium & 790€/Tag \\
\hline
Tim Tiek & Test Manager & Extern & 10 Jahre & Zertifiziert & Viel Erfahrung mit HP Tests & 860€/Tag \\
\hline
Tom Tonk & Software Tester & Extern & Einige Jahre & Zertifiziert & Internatio-nale Projekte & 760€/Tag \\
\hline
\end{tabularx}
\end{center}

\begin{enumerate}
  \item \textbf{Qualität}:
    Die Erfahrung wird wie folgt bewertet:
    \begin{itemize}
      \item 1 Jahr: Sehr negativ
      \item 2-3 Jahre: Negativ
      \item Über 3 Jahre: Positiv
      \item Über 5 Jahre: Sehr positiv
    \end{itemize}

  \item \textbf{Arbeitsmoral}:
    Die Arbeitsmoral ist als potentieller Gegenspieler zur Erfahrung zu werten, da ältere Arbeitnehmer in der Regel nicht mehr die selben Aufstiegsambitionen wie junge Arbeitnehmer haben. Eine hohe Verfügbarkeit wirkt sich positiv aus.

  \item \textbf{Diversität}:
    Die vorherigen Rollen haben bereits vieles abgedeckt, daher besteht kein akuter Handlungsbedarf. Kreative Hobbies, Auslandserfahrung und Nischenwissen wirken sich jedoch positiv aus.

  \item \textbf{Kosten}:
    Die Kosten werden wie zuvor bewertet.
    
\end{enumerate}

\begin{center}
\small
\begin{tabularx}{\textwidth}{|l|X|X|X|X|X|X|}
\hline
\textbf{Name} & \textbf{Qualität} & \textbf{Arbeits-moral} & \textbf{Diversität} & \textbf{Kosten} & \textbf{Ergebnis} \\
\hline
Harry Mayer & - - & ++ & + & ++& (+++) \\
\hline
Stefan Schmitt & - & ++ & ++ & ++& (+++++) \\
\hline
Tom Schulze & + & + & ++ & ++&  (++++++) \\
\hline
Sam Gamdschie & ++ & - - & ++ & - & (+) \\
\hline
Rubeus Hagrid & - & + & + & + & (++) \\
\hline
Tim Tiek & ++ & - & ++ & - & (++) \\
\hline
Tom Tonk & - & ++ & ++ & + & (++++) \\
\hline
\end{tabularx}
\end{center}
Es könnten zwischenmenschliche Probleme bei Harry Mayer und Tom Schulze entstehen, da ersterer gerne alleine arbeitet und disruptive Methoden anwendet, während letzterer viel Wert auf Erfahrungsaustausch und Zusammenarbeit legt. Trotzdem werden beide aufgenommen, da sich Harry Mayer mit seiner Art sehr gut mit dem laissez-faire-Stil des Product Owners kombinieren lässt, und auf der anderen Seite Tom Schulze mit den kommunikativen Scrum Mastern gut zurechtkommen wird.

\textit{\textbf{- Harry Mayer, Stefan Schmitt und Tom Schulze sind die präferierten Entwickler. Tom Tonk ist der gewählte Tester.}}
\vspace*{0.1cm} \\
\textbf{Marketing-Kauffrau}: Trägt dazu bei, die Sichtbarkeit des Projekts und seiner Ergebnisse zu erhöhen, um Akzeptanz und Unterstützung zu fördern. Hierbei stehen zwei Kandidatinnen zur Auswahl.
\begin{center}
\small
\begin{tabularx}{\textwidth}{|>{\arraybackslash}p{2.2cm}|X|>{\arraybackslash}p{.9cm}|X|>{\arraybackslash}p{1.6cm}|>{\arraybackslash}p{2.1cm}|>{\arraybackslash}p{1.6cm}|}
\hline
\textbf{Name} & \textbf{Rolle} & \textbf{Typ} & \textbf{Erfahrung} & \textbf{Verfügbar} & \textbf{Besonders} & \textbf{Kosten} \\
\hline
Mira Bellenbaum & Marketing-Kauffrau & Intern & Neu & 100\% & Egozen-trisch & 520€/Tag \\
\hline
Maria Musterfrau & Marketing-Kauffrau & Extern & Viel Erfahrung & Stundenba-sis & Egozen-trisch & 100€/Stun-de \\
\hline
\end{tabularx}
\end{center}
Da die Vermarktung nur vereinzelt stattfinden soll, bietet sich Maria Musterfrau besser an, da hier eine flexiblere stündliche Bezahlung erfolgen kann und somit weniger Kosten anfallen.

\textit{\textbf{- Maria Musterfrau übernimmt die Marketing Kampagne sobald diese benötigt wird.}} \\

\section{Finale Teamauswahl}

\begin{figure}[h]
	\centering
	\includegraphics[scale=1.2]{img/TEAM.pdf}
	\captionsetup{format=hang}
	\caption{\label{fig:Scrum}Darstellung des finalen Teams}
\end{figure}

Das finale Scrum-Team besteht aus insgesamt sieben Personen. Vervollständigt wird die Grafik durch das unterstützende Marketing. Es gibt folgende Abweichungen zu dem zuvor skizzierten optimalen Team:
\begin{itemize}
\item Es gibt zwei Scrum Master, die beide jeweils zu 50\% angestellt sind. Dies könnte sich nachteilig auf die Kommunikation und die Verantwortlichkeiten der beiden auswirken. Nichtsdestotrotz bietet das Konstrukt auch einige Vorteile: Erstens kann so die interne Verantwortlichkeit garantiert werden, und darüber hinaus kann auch die Diversität gesteigert werden, da zwei zusätzliche Frauen im Team sind. Zum Schluss handelt es sich bei den Kolleginnen um sehr angesehene, gut kommunizierende und strukturierte Mitarbeiterinnen, die als gewisser Ausgleich zum Führungsstil des Product Owners fungieren.
\item Die Entwicklung wurde nicht wie geplant ausgelagert, sondern befindet sich vollständig in-house. Das ist aufgrund der Bewertungskriterien die bessere Wahl gewesen, insbesondere da es nicht genügend externe Backend-Entwickler gibt.
\end{itemize}

Damit ist das Team vollständig, und im nächsten Schritt können basierend auf den Arbeitspaketen die Kosten für das Projekt bestimmt werden.

\section{Gesamtkosten}

Sachliche Hilfsmittel sind allesamt vorhanden und werden dementsprechend im folgenden nicht berücksichtigt. Der Umrechnungsschlüssel der Punkte im Planungspoker ist 1:1 in Arbeitstage umgerechnet. Alle ungeraden Nummern werden auf ganze Tage aufgerundet. Zudem handelt es sich bei folgender Zuteilung nicht um die Verantwortlichkeiten der Arbeitspakete, sondern um die daran mitarbeitenden Personen.

\begin{center}
\small
\begin{longtable}{|>{\arraybackslash}p{1.5cm}|>{\arraybackslash}p{2.2cm}|>{\arraybackslash}p{2cm}|>{\arraybackslash}p{3.8cm}|>{\arraybackslash}p{1.5cm}|>{\arraybackslash}p{1.4cm}|}
\hline
\textbf{Arbeits-paket} & \textbf{Aufgabe} & \textbf{Dauer} & \textbf{Zuständig} & \textbf{Kosten} &  \textbf{Gesamt-kosten}\\
\hline
\multicolumn{6}{|l|}{\textbf{Konzeptionierung (ges. 13.450€)}} \\
\hline
AP-K-1 & IT-Architekt & 7 Tage & Aufgabe von Scrum Master Anna Schmidt & 550€ pro Tag & 3.850€ \\
\hline
AP-K-2 & UX-Designer & 15 Tage & Aufgabe von Scrum Master Susi Sonnenschein & 640€ pro Tag & 9.600€ \\
\hline
\multicolumn{6}{|l|}{\textbf{Entwicklung (ges. 42.600€)}} \\
\hline
AP-E-1 & Backend Entwickler & 13 Tage & Aufgabe von Entwickler Harry Mayer & 600€ pro Tag & 7.800€ \\
\hline
AP-E-2 & Backend Entwickler & 13 Tage & Aufgabe von Entwickler Tom Schulze & 600€ pro Tag & 7.800€ \\
\hline
AP-E-3 & Backend Entwickler & 20 Tage & Aufgabe von Entwickler Harry Mayer & 600€ pro Tag & 12.000€ \\
\hline
AP-E-4 & Frontend Entwickler & 25 Tage & Aufgabe von Entwickler Stefan Schmitt & 600€ pro Tag & 15.000€ \\
\hline
\multicolumn{5}{|l|}{\textbf{Testing (ges. 18.520€)}} \\
\hline
AP-T-1 & Tester & 7 Tage & Aufgabe von Tester Tom Tonk & 760€ pro Tag & 5.320€ \\
\hline
AP-T-2 & Front-Backend-Entwickler & 11 Tage & Aufgabe von Entwickler Tom Schulze und Stefan Schmitt zu gleichen Anteilen & 1.200€ pro Tag & 13.200€ \\
\hline
\multicolumn{6}{|l|}{\textbf{Pilotierung (ges. 51.600€)}} \\
\hline
AP-P-1 & Product Owner & 40 Tage & Aufgabe von PO Franz Urlaub & 830€ pro Tag & 33.200€ \\
\hline
AP-P-2 & Business Analyst, Front-Backend-Entwickler & 10 Tage & Aufgabe von Scrum Masterin Susi Sonnenschein, Stefan Schmitt, Harry Mayer & 1.840€ pro Tag & 18.400€ \\
\hline
\multicolumn{6}{|l|}{\textbf{Rollout (ges. 46.480€)}} \\
\hline
AP-R-1 & Marketing, Business Analyst & 8 Tage & Aufgabe von Marketing Maria Musterfrau in Kooperation mit Susi Sonnenschein & 1.440€ pro Tag & 11.520€ \\
\hline
AP-R-2 & Business Analyst, Front-Backend-Entwickler & 19 Tage & Aufgabe von Scrum Masterin Susi Sonnenschein, Stefan Schmitt, Harry Mayer & 1.840€ pro Tag & 34.960€ \\
\hline
\multicolumn{6}{|l|}{\textbf{Gesamtprojekt (ges. 172.650€)}} \\
\hline
\end{longtable}
\end{center}

In dieser Tabelle werden sowohl die Kosten für die zugeteilten Mitarbeiter pro Arbeitspaket als auch deren kombinierte Tagessätze aufgeführt. Multipliziert man diese Werte, erhält man die Gesamtkosten pro Arbeitspaket. Akkumuliert man diese Kosten, erhält man die Gesamtkosten pro Phase, und wenn man sie noch einmal summiert, erhält man die Gesamtkosten für das Projekt, basierend auf den Arbeitspaketen. Die Rollenaufteilung funktioniert besonders gut bei den Entwicklungsaufgaben, da hier klar spezifiziert ist, welche Aufgaben erledigt werden müssen. Komplizierter wird es bei Aufgaben, die allgemein von Wirtschaftsinformatikern erledigt werden müssen. Hier kommt häufig die Business Analystin zum Einsatz, um den Product Owner zu unterstützen, da dieser bereits sehr große Arbeitspakete hat. Es wurde darauf geachtet, dass in den gleichen Phasen so viele Pakete wie möglich parallelisiert werden, um die Projektlaufzeit zu reduzieren.

Die kalkulierte Summe unterscheidet sich substantiell von der Milchmädchenrechnung der Projektkosten, bei der man einfach die Tagessätze des gesamten Teams nimmt und auf die anfängliche angenommene Projektlaufzeit multipliziert. Die summierten Tagessätze belaufen sich auf 4.785€ pro Tag, und ein Jahr hat ungefähr 250 Arbeitstage. Hierbei kommt man auf einen angenommenen Wert von:
\begin{center}
\textbf{FALSCH:}\\
SUM(TAGESSATZ) * ARBEITSTAGE = 4.785€ * 250 = 1.196.250€\\
\textbf{RICHTIG:}\\
ARBEITSPAKETE * KOSTEN = 172.650€\\
\end{center}
Dieser Ansatz führt zu ungefähr dem siebenfachen der zuerst errechneten Gesamtsumme. Das wäre eine erhebliche Ungenauigkeit in der Planung und unterstreicht die Relevanz von genau definierten Arbeitspaketen, durch Methoden wie das Planning Poker. 

\begin{figure}[h]
\centering
\includegraphics[scale=.2]{img/COST.pdf}
\captionsetup{format=hang}
\caption{\label{fig:COST}Darstellung der Arbeitspaket-Kosten}
\end{figure}

Insgesamt lässt sich an diesem Graphen sehen, dass vor allem die drei Schritte Entwicklung, Pilotierung und Rollout am kostenintensivsten sind. Die Konzeption und das Testing sind dagegen kleinere Schritte und dementsprechend auch weniger kostspielig. Mit dieser finalen Gesamtkostenplanung ist die Ressourcenplanung abgeschlossen und es resultiert ein potentielles Scrum-Team sowie dessen veranschlagte Kosten.

\singlespacing
\initializeBibliography
\end{document}

